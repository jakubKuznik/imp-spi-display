Cílem projektu je zapojit a zprovoznit  TFT (thin-film transistor) displej, který komunikuje přes rozhraní SPI (Serial Peripheral Interface) na zařízení ESP32 a demonstrovat jeho funkčnost pomocí Wifi či Bluethoot. V našem případe budeme demonstrovat funkčnost displeje pomocí Wifi a to tak, že na ESP32 poběží webový server s jednoduchou webovou stránkou přes, kterou můžeme změnit aktuální zobrazení displeje. Pro připojení na webový server zařízení ESP32 rovněž vystupuje v roli AP (Access Point).


% Výběrová kritéria podle nichž se rozhoduje závisí na poměru obranných a útočných systémů a ideálním poměrem balistických vlastností k ceně. Cílem této práce je zjistit, které zbraně a obranné systémy se dvou různým stranám vyplatí nakupovat a jaký je optimální poměr nákupu obranných nebo útočných systémů. 



%\todo{
%Jak na to:
%1. Co chceme zjistit/ukazat/vyvratit (vstupni otazka) \newline
%    - plati, neplati, podporit, vyvratit \newline
%    - z toho by melo vyplynout co do toho modelu patri a co uz ne !! \newline
%}
%\newline
%\todo{
%Co řešíme? 
%proč to řešíme? 
%jak se to řeší? 
%}
%
%\todo{v prvnich nekolika vetach rict co zkoumame a k cemu chceme dojit }



%V současném světě kde téma války je denním pořádku nás napadlo jaká je nejvhodnější strategie pro nákup zbrani a obranných systému. jaké zbraně jsou nejúčinější a tudíž se je vyplatí použít. Jaký je nejlepší poměr pro nákupu zbraní tak, aby strana co nejdřív zvítězila. Simulace se opírá o hlavní parametry zbraní a obranných systému. Jako je ničivost střel, dostřel, počet setřel možných vystřelených za minutu, cena za zbraně i její vystřelenou munici.